\input{preamble}

%----------------------------------------------------------------------------------------
%	TITLE SECTION
%----------------------------------------------------------------------------------------

\title{\vspace{-15mm}\fontsize{24pt}{10pt}\selectfont\textbf{Robot Project}} % Article title

\author{
\large
\textsc{Rune A. Heick 11061}\\
\textsc{René Arendt Sørensen 11553} \\ 
\textsc{Nicolai Glud 20093625}\\[2mm] % Your name
\normalsize Aarhus University, Department of Engineering \\ % Your institution
\vspace{-5mm}
}
\date{}

%----------------------------------------------------------------------------------------

\begin{document}
\setlength{\abovedisplayskip}{1cm}
\setlength{\belowdisplayskip}{.8cm}
\maketitle % Insert title

\newpage
%----------------------------------------------------------------------------------------
%	ABSTRACT
%----------------------------------------------------------------------------------------

\begin{abstract}
The Content of this paper seeks to present the knowledge gained throughout the AI in Robotics and kalman filters reading course from Aarhus University, department of engineering. The paper is split into two larger sections,  theory and project. The topics of the reading course are explained in brief terms in the theory section.  A self driving robot that are capable of localizing it self and follow a path has been implemented In the project section. A discussion of the various issues found when working on the project can be found in the Discussion section and lastly we conclude on a successful project.
\end{abstract}
\tableofcontents

%----------------------------------------------------------------------------------------
%	ARTICLE CONTENTS
%----------------------------------------------------------------------------------------

%\begin{multicols}{2} % Two-column layout throughout the main article text
\chapter{Introduction}
The goal of this report is researching how to implement AI in Robotics techniques. The knowledge required for this is acquired from the Udacity course "Artificial Intelligence for Robotics"\cite{AIROK}. The book Probabilistic Robots will be used to supplement the knowledge from the Udacity course\cite{thrun2005probabilistic}.

The overall purpose of the course can be described as these points:
\begin{itemize}
\item Define and explain concepts, methods and technologies relating to the chosen subject area.
\item Give an account of research articles relevant to the subject area.
\item Account for the status and application of the subject area.
\item Prepare a report and an oral presentation on the subject area.
\end{itemize}

To fulfil these points this report will be split into two parts, theory and project. The theory will seek to provide an insight into the gained theoretically knowledge about artificial intelligence in robots.
The project part will entail the gained knowledge of implementing the theory in a moving robot.
%------------------------------------------------
% chapter theory
\chapter{Theory}
\section{Localisation}
Localisation is the act of a robot finding out where it is. GPS systems are often used to do this in cases where precision is not critical, e.g. GPS navigation. If precision is critical, e.g. for a robot to drive past obstacles without hitting them, GPS is too imprecise. Other techniques must then be used. To do this, robots often have sensors, such as sonars, LIDARs or cameras. 
The localisation algorithms consists of two main parts, movement and measurement. Movement is imprecise, often because of the mechanical system. Measurements contributes with new knowledge and is therefore the part that narrows the decision, of where the robot is, in. The measurements is often also noisy, but usually less noisy than the movements.
Another important thing in localisation is probability. What is the probability of being in a certain position, given the measurement? To calculate this, Bayes rule is used.
\begin{equation}
p(x|z) = \frac{p(z|x)\cdot p(x)}{p(z)}
\end{equation}
Where $\frac{1}{p(z)}$ is the normalizing factor $\eta$.
\begin{equation}
\eta = \frac{1}{p(z)} = \frac{1}{\sum(p(z|x)\cdot p(x))}
\end{equation}
This gives us the expression:
\begin{equation}
p(x|z) = \eta \cdot p(z|x)\cdot p(x)
\end{equation}
\textbf{Example:}\\
A robot is in one of the grids in the map. It can detect if it is in a red or green grid. 
\begin{figure}[H]
\centering
\includegraphics[scale=1]{billeder/Localisation01.png}
\caption{The map in which the robot lives}
\label{fig:Localisation01}
\end{figure}
If the robot is in a green grid, it detects correctly with 75\% chance. If it is in a red grid, it detects correctly with 90\% chance.
If the robot detects green, what is the probability that it is in a green or a red grid?\\
$z = green$\\
$\eta = \frac{1}{sum(p(z|x)\cdot p(x))} = \frac{1}{0.75 \cdot \frac{3}{5} + 0.1 \cdot 2/5} = 2.04$\\\\
$x = green$\\
$p(x) = \frac{3}{5}$\\
$p(x \mid z) = \eta \cdot p(z|x)\cdot p(x) = 2.04 \cdot 0.75 \cdot \frac{3}{5} = 0.918 = 91.8\%$\\\\
$x = red$\\
$p(x) = \frac{2}{5}$\\
$p(x \mid z) = \eta \cdot p(z|x)\cdot p(x) = 2.04 \cdot 0.1 \cdot \frac{2}{5} = 0.082 = 8.2\%$
\subsection{Markov Localisation}
% Localisation_MarkovLocalisation
The purpose of Markov Localisation is to estimate where the robot is located, given its measurements, movements and a map describing the world in which the robot is placed.
Lets say the robot can move along a wall and sense whether if there is a door next to it or not. We place the robot in a world with a wall with 3 doors.
The estimation of the localisation of the robot is called belief and is the probability of being in a given location, given the previous act. At the beginning, the robot will not know where it is, and therefore it will have a uniformly distributed belief.\\ 
Then the robot takes a measurement, and let's say that it measures a door. Now the new belief is the former belief multiplied with the probability of the measurement z given the position x. 
\begin{equation}
bel'(x) = bel(x) \cdot p(z|x)
\label{ML_eq1}
\end{equation}
After this the robot moves. A movement is often noisy and therefore the precision of the belief will fall. This is illustrated by convolving the belief with the motion model. The motion model consists of a mean $\mu$, representing the noise free movement, and a standard deviation $\sigma$, representing the noise. The result of this step can be seen in \ref{ML_fig1:sub3}.
\begin{equation}
bel'(x) = bel(x) \ast norm(\mu,\sigma)
\label{ML_eq2}
\end{equation}
These two steps are now repeated the rest of the lifetime of the robot. Every time the robot senses, it gets more certain of where it is and every time it moves it gets more uncertain. In figure \ref{ML_fig1} two iterations are shown.

\begin{figure}[H]
\centering

\begin{subfigure}[b]{.8\textwidth}
  \centering
  \includegraphics[width=1\linewidth]{billeder/MarkovLocalisation01.png}
  \caption{Initial belief is uniformly distributed.}
  \label{ML_fig1:sub1}
\end{subfigure}%

\begin{subfigure}[b]{.8\textwidth}
  \centering
  \includegraphics[width=1\linewidth]{billeder/MarkovLocalisation02.png}
  \caption{Measurement and result of belief multiplied with measurement.}
  \label{ML_fig1:sub2}
\end{subfigure}

\begin{subfigure}[b]{.8\textwidth}
  \centering
  \includegraphics[width=1\linewidth]{billeder/MarkovLocalisation03.png}
  \caption{Movement of robot and belief. Noise makes the belief after the movement more uncertain.}
  \label{ML_fig1:sub3}
\end{subfigure}%

\begin{subfigure}[b]{.8\textwidth}
  \centering
  \includegraphics[width=1\linewidth]{billeder/MarkovLocalisation04.png}
  \caption{Measurement and result of belief multiplied with measurement.}
  \label{ML_fig1:sub4}
\end{subfigure}

\begin{subfigure}[b]{.8\textwidth}
  \centering
  \includegraphics[width=1\linewidth]{billeder/MarkovLocalisation05.png}
  \caption{Movement of robot and belief. Noise makes the belief after the movement more uncertain.}
  \label{ML_fig1:sub5}
\end{subfigure}

\caption{Markov Localisation illustrated. From Probabilistic Robotics by Sebastian Thrun, Wolfram Burgard and Dieter Fox.}
\label{ML_fig1}
\end{figure}

The algorithm for Markov Localisation is as follows:\\
1:\quad \textbf{Algorithm MarkovLocalization(}$bel(x_{t-1}),u_t,z_t,m$\textbf{):}\\
2:\quad \quad $for\,all\,x_t\,do$\\
3:\quad \qquad $\overline{bel}(x_t) = \int p(x_t|u_t,x_{t-1},m)\cdot bel(x_{t-1}) dx$\\
4:\quad \qquad $bel(x_t) = \eta p(z_t|x_t,m)\cdot \overline{bel}(x_t)$\\
5:\quad \quad $endfor$\\
6:\quad \quad $return\,bel(x_t)$

What happens is that after a move, $u_t$, and a measurement, $z_t$, has been taken, the algorithm is called. First it calculates the belief after the movement, and then it calculates the belief after the measurement. It does this for all possible places, $x_t$, in the map, $m$. This corresponds to the combinations of (a and b) or (c and d) in figure \ref{ML_fig1}.
\subsection{Kalman Filter}
The kalman filter is a method for sensor fusion. It is capable off taking two or more measurements and fuse together to one optimal guess of the true value. It does this by looking at the current measurement and compare it with a predicted value. The predicted value is often a combination of the previous value combined with the changed done since the last measurement. 
\\\\
The kalman filter has some limitations. One of them is that it assume that all noise in the system is gaussian distributed, which is not always the case. Another is that it all ways gives only one guess on the value, which is optimal for a lot of systems but is not especially preferable in localisation. The last one is that it assumes that there is a linear relationship between all measurements and the value we are trying to estimate. 
\\\\
Much like the markov localisation the kalman filter have a mesurement step, then it does some movement and predict the new value. This is done in a never ending continues process. 

\begin{figure}[H]
\includegraphics[scale=0.51]{billeder/KalmanFilter.png}
\end{figure}

If we look at the math, we can see that this can be done in a few steps. looking at the equations of the kalman filter we normally call 2-3 for the prediction step and 4-7 for the measurement step. 

\subsubsection{Prediction Step}
Looking at the equations in the prediction step (\ref{Pre1}, \ref{Pre2}).
\begin{equation}
\overline{\mu}_t = A_t \mu_{t-1} + B_t u_t
\label{Pre1}
\end{equation}
\begin{equation}
\overline{\Sigma}_t = A_t \Sigma_{t-1} A^{\intercal}_t+ R_t
\label{Pre2}
\end{equation}

There are 2 conversion matrices $A$, $B$ and a measurement/input matrix $u$. The matrix $A$ must transform the last best estimate to the current domain. e.g. if we are trying to predict how fare down a bungee jumper have fallen then the $A$ matrix must add the the distance the gravity would have pulled him down since the last estimate $\mu_{t-1}$. We call this the static influence, since it is the change in the system if we do not interfere. The input matrix $u$ is the change we put in to the system since the last prediction. Lets say that we can increase or decrease the bungee jumpers air resistance. The $u$ matrix will contain the information about the change in air resistance. The $B$ matrix will convert the input matrix to the desired domain. For the bungee jumper it will convert air resistance to a fall distance. We call this the dynamic influence.
The last matrix $R$ is the noise, expressed as the covariance, of the input matrix $u$. This is to inform the system about the certainty of the input. 
\\\\
As previously mentioned this dictates that there exists a linear conversion from the static and dynamic influences in relation to the the previously state to the new state. 

\subsubsection{Measurement Step}
Looking at the equations in the measurement step (\ref{Mes1}, \ref{Mes2} , \ref{Mes3}).
\begin{equation}
K_t = \overline{\Sigma}_t C^{\intercal}_t ( C_t \overline{\Sigma}_t C^{\intercal}_t + Q_t ) ^{-1}
\label{Mes1}
\end{equation}
\begin{equation}
\mu_{t} = \overline{\mu}_t +K_t (z_t-C_t\overline{\mu}_t)
\label{Mes2}
\end{equation}
\begin{equation}
\Sigma_t = (I-K_t C_t) \overline{\Sigma}_t
\label{Mes3}
\end{equation}

The measurement step takes the predicted value $\overline{\mu}_t$, $\overline{\Sigma}_t$ and uses the measurement to fix the error in the prediction introduced by the noise. The measurement vector $z_t$ contains the measurement. For the bungee jumper it could be the tension in the bungee line. The conversion matrix $C_t$ convert the measurement to the value domain. So it takes the tension and convert to a distance. The $Q_t$ matrix is our measurement noise, expressed as the covariance.
\\\\
By calculating the Kalman gain $K_t$ we get a kind of believe on the measurement, that decides how much we are able to affected the predicted value with information obtained from the measurement. 

After we have corrected the prediction, we have the best possible guess of the true value, or the true distance in the bungee case. 

\subsubsection{Graphical}
Looking at the bungee example, this can also be graphically shown. On figure \ref{bungeeFig} we see the bungee problem. 

\begin{figure}[H]
\includegraphics[scale=0.60]{billeder/Bungee.jpg}
\caption{The bungee problem}
\label{bungeeFig}
\end{figure}

First we see a the prediction step which have a very high uncertainty as shown with a Gaussian with a big variance. After this we make a measurement, as show in the second graph. we now use the kalman filter to eastimate the best guess, based on the prediction and the measurement. 

We can see this as optimal fusion of our prediction information and our measurement information. 

\subsubsection{Kalman Filters In Robotics}
Like in the bungee example were the position of the bungee jumper was found. A robot can be located in a known map. But this is not only use of a kalman filters in robotics. \\
kalman filters is often used to fuse multiple sensor readings to one optimal guess, and to discover hidden variables. A hidden variable is a property that can not be directly measured, but have a relation to the measured data. lets say you can measure the location of a robot. Then the hidden variable can be the robots speed since the speed can be found looking at different locations over time. By using a kalman filter it is possible to eastimate the speed very precise. \\
Since the kalman filter is uni-model, which mean that it only can give one guess of the location. Other techniques like particle filters and markov localisation that give multiple candidate locations are often preferred. 

\subsubsection{Extended Kalman Filter}
One of the big limitations of the kalman filter is the requirement that dictates that there must be a linear relationship between measurements or input and the value domain. 

This limitation is removed in the extended kalman filter. This is done by replacing the linear conversion matrices $A$, $B$ and $C$ with the none linear functions $g(u_t, \mu_{t-1})$ and $h(\overline{\mu}_t)$.\\

Since the kalman filter basically exploits the principle of a Gaussian converted with a linear transformation still is a Gaussian. Introducing the non linear functions brakes the filter. To fix this the extended kalman filter utilize a method called first order Taylor expansion. Taylor expansion construct a linear approximation to a function, in a given point. This is done by using the partial derivative of the function in the point. We create the partial derivative matrices $G$ and $H$.

\begin{equation}
G = \frac{\partial g(u_t, \mu_{t-1})}{\partial \mu_{t-1}} \\ H = \frac{\partial h(\overline{\mu}_t)}{\partial \overline{\mu}}
\label{Mes3}
\end{equation}

We can now use this in the regular kalman filter to obtain the extended kalman filter.

\begin{figure}[H]
\includegraphics[scale=0.51]{billeder/EKF.png}
\end{figure}

It have been shown that this type of kalman filter is far better in particle that the regular kalman filter if the variance of prediction and measurement is small. If this is not the case the linearisation introduces additional noise. This is due to the error between the linearisation and the true function. The greater the variance the greater impact does this error have.  

\subsection{Particle Filter}
The particle filter is a different kind of filter, since it uses a discrete method for finding the robots location, in contrast to the continuously methods previously discuses. \\
The method is multi-model, which means that more than one location candidate can be found. This is a very popular technique for localisation, since it is simple to understand and implement. 

Like the other filters, the particle filter does also require the map to be known, and we must be able to predict how all measurements will look in any given location. 

The particle filter can also be seen as having a prediction step and a measurement step. Before we start the measurement-prediction loop, we first select a number $M$ of particles $X$ to use in the filter. Then we place all the $M$ particles at random locations. 

\subsubsection{Prediction Step}
After the robot have moved with a motion $u_t$, we move all the particles with the same motion. On this motion we add some random noise $w_t$. If one of the particles was exactly at the robot location before we moved it, and the noise we put on the particle was the exact same as on the real robot, the new particle location would still be the same as the robot.  

\subsubsection{Measurement Step}
First we take the measurement from the real robot $z_t$. Then we compare it to all of the $M$ particles and calculate if the probability of the location of the real robot is the same as for the particle. This is done by comparing the measurement $z_t$, with the measurement that would have been at the particle location.
In this process it is important to assume that there are noise on the measurements. If we assume that the sensor is too precise we risk giving particles that are close to the real location a low probability, which is not desirable.

When we have all the probabilities for the particles $Prob_t$, we are doing resampling. The aim of resampling is to get a new set of $M$ new particles. The locations of the new particles must be randomly drawn from the old particle set with a distribution that matches the $Prob_t$ distribution. This means that a lot of the new particles are at the locations with high probability, where only a few or none is at locations with low probability.

After the resampling we continue with the prediction step, and so one. 

\subsubsection{The Algorithm}
Over time particles with at bad locations will die, and more will search the area around feasible locations. The predict - measurement steps can also be expressed as pseudo code as seen below: 
\begin{figure}[H]
\centering
\includegraphics[width=0.8\textwidth]{billeder/ParticleFilter.png}
\caption{Particle Filter Algorithm}
\label{fig:ParticleFilter}
\end{figure}
This process runs continuously, and will over time converge to zero or one single location.

\subsubsection{Particle Filters Considerations}
A series of scenarios must be considered when using particle filters. One is the scenario where there is not placed a particle close to the real robot or all the particles around it dies. This will make the particle filter coverage around a wrong location or no location. There are several techniques tackling this problem. One easy simple solution is to look at probability and see if there is at least one good candidate. If this is not the case the filter should be reset. 

One other problem happens if the world is symmetric, this means that every measurement have multiple feasible locations, and when moved previously feasible locations still appearers equally likely. If this is the case it is impossible to know the true location. This can also in extreme cases be a problem if the world is only partially symmetric, since all the true particles can die, and only the wrong survive. When we now enter a asymmetric area all the wrong will die, and we have no particles left. This can be fixed by either resetting the filter, or by a technique where you always randomly spread out the worst $5\%$ of you particles. 

Also a balance must be found with the number of particles and the map size. The more particles the better the filter, but it will also drastically increase the calculation time. 

%------------------------------------------------
\section{Robot Motion}
This section will seek to explain three different motion models. The velocity based motion model, the odometry based motion model and the map based motion model. The models acts as a way to determine the outcome of a movement command. It is based on a robot pose which can be seen in figure \ref{fig:robotpose}.
\begin{figure}[H]
\centering
\includegraphics[width=0.8\textwidth]{billeder/robotpose}
\caption{Robot pose}
\label{fig:robotpose}
\end{figure}
The pose is defined as a vector in a 2D world:
\begin{equation}
robotpose := ( x, y, \theta ) ^T
\end{equation}
The vector has a position, $[ x, y ]$ and and an angle, $\theta$. For robots moving in more dimensions, the vector will have to be larger to accommodate the coupling between these dimensions. Only 2D cases will be explained in this section.\\

When a robot moves, the movement will not produce a perfect transition between two positions. The intrinsic noise in the movement apparatus will provide both angular noise and positional noise. This means our model will have to be probabilistic. The probability of being in position $x_t$ given a motion control command $u_t$ and a previous position is defined as in equation \ref{eq:posterior}.
\begin{equation}
p(x_t | u_t, x_{t-1})
\label{eq:posterior}
\end{equation}
An example of posterior distribution is seen in figure \ref{fig:postdist}. The figure shows that more complex movement behaviours will result in more uncertainty of where the robot is positioned when the movement ends.
\begin{figure}[H]
\centering
\includegraphics[width=0.8\textwidth]{billeder/postdist}
\caption{Posterior distribution}
\label{fig:postdist}
\end{figure}

\subsection{Velocity based motion model}
In the velocity based motion model, the posterior distribution is still given by equation \ref{eq:posterior}. The control command is defined as $u_t = (v_t, \omega_t )^T $. $v_t$ is the translational velocity at time t. $\omega_t$ is the rotational velocity at time t. In figure \ref{fig:velocmod} three different cases are seen. (a) is a regular case with moderate translation and rotational error. (b) is a case with large translational error and (c) is a case with large rotational error.
\begin{figure}[H]
\centering
\includegraphics[width=0.8\textwidth]{billeder/velocmod}
\caption{Velocity based motion model}
\label{fig:velocmod}
\end{figure}
The algorithm for the velocity based motion model can be seen in figure \ref{fig:motionmodelalgo}. This is the algorithm behind the model seen in figure \ref{fig:velocmod} as well. The \textbf{prob} function is either normal distributed or triangular distributed. $\alpha_1$ to $\alpha_6$ determines the error introduced in our model.
\begin{figure}[H]
\centering
\includegraphics[width=0.8\textwidth]{billeder/motionmodelalgo}
\caption{Velocity based motion model algorithm}
\label{fig:motionmodelalgo}
\end{figure}
When executing the algorithm the outcome will be an estimated final pose. This is useful as it can be used for prediction purposes. For discrete spaces sampling velocity based motion model algorithm is used. This is seen in figure \ref{fig:samplemodelalgo}.
\begin{figure}[H]
\centering
\includegraphics[width=0.8\textwidth]{billeder/samplemodelalgo}
\caption{Sampling velocity based motion model algorithm}
\label{fig:samplemodelalgo}
\end{figure}
The \textbf{sample} function returns a sample from either a normal or a triangular distribution with $\mu = 0$ and $\sigma^2$ being the input parameter. As with the non-sampling version, $\alpha_1$ to $\alpha_6$ determines the error introduced in our model.
\subsection{Odometry based motion model}
The odometry based motion model uses measurements in order to calculate the robot position. This is usually done using information from the wheels, the so called wheel encoders, and integrating it over time. Most platform have this implemented in hardware and lower level drivers. This method is usually more accurate than the velocity based motion model. One big point is that odometry is only useful after the robot has moved.  This means we can not use the information to accurately plan out our motion. In figure \ref{fig:odometry} it is seen that odometry uses a rotation, followed by a translation and a second rotation to estimate the motion.
\begin{figure}[H]
\centering
\includegraphics[width=0.6\textwidth]{billeder/odometry}
\caption{Odometry based motion model}
\label{fig:odometry}
\end{figure}
The algorithm for odometry is seen in figure \ref{fig:odometryalgo}. It fits neatly with the way odometry is seen in figure \ref{fig:odometry}.
\begin{figure}[H]
\centering
\includegraphics[width=0.8\textwidth]{billeder/odometryalgo}
\caption{Odometry based motion model algorithm}
\label{fig:odometryalgo}
\end{figure}


\subsection{Map based motion model}
Map based motion models seek to incorporate a map in the posterior distribution seen in figure \ref{eq:posterior}. This is done when the map contains relevant information that leads to the following equation:
\begin{equation}
p(x_t | u_t,x_{t-1}) \neq p(x_t | u_t,x_{t-1},m)
\end{equation}
Calculating the map based motion model is a complex task as it needs to incorporate the probability of a path existing in the first place. In order to simplify, we assume that the distance between $x_t$ and $x_{t-1}$ is smaller than half  of the robots diameter. The approximation becomes:
\begin{equation}
p(x_t | u_t,x_{t-1},m) = \eta p(x_t | u_t,x_{t-1})p(x_t|m)
\end{equation}
$\eta$ is the normaliser and $p(x_t|m)$ tells us if it is possible for the robot to even be in that location. An example is seen in figure \ref{fig:mapbased}. For (a)  we assume that we end up somewhere along the black arch. What we do not know is that there is a wall there. When we incorporate the map in (b) we see that for all areas in and around the wall, $p(x_t|m) = 0$.
\begin{figure}[H]
\centering
\includegraphics[width=0.8\textwidth]{billeder/mapbased}
\caption{Map based motion model}
\label{fig:mapbased}
\end{figure}

\section{Search and Planning}
The purpose of search and planning is to find the optimal path to the goal. This is done in a discrete environment where you have multiple factors in play. Some of these factors are map, locations \& obstacles and cost. Locations \& obstacles are placed in the map according to their position in the real world. These items will stand in way of our route planning, so we will have to find a way around them. This is illustrated as a matrix below:
\[
\begin{bmatrix}
S & 1 & 0 & 0 & 0\\ 
0 & 1 & 0 & 0 & 0\\ 
0 & 1 & 0 & 0 & 0\\ 
0 & 0 & 0 & 1 & 0\\ 
0 & 0 & 0 & 1 & G
\end{bmatrix}
\]
Ones correspond to obstacles and zeros to cells where we can traverse. S is the start position and G is the Goal.\\
Searching for a path to the goal is done by expanding the cells around the current position. Whenever we expand a cell we also add the cost of moving to that cell to our overall cost value. When we have found our goal, the path with the lowest cost value will be our preferred path.\\
Another example of cost could be if you have a vehicle and were driving through rush hour. The cost of taking left turns would be higher than right turns. This is incorporated into the cost function when searching for a path.\\

\subsection{A\text{*}}
A\text{*} or A star is way to find the shortest path without expanding every cell in our map. This is done with a Heuristic function (called h). The value of the cells in the Heuristic function is added to the cost when expanding cells, $f = h + g$ with g being the cost. The path with the lowest f value is our preferred path. There is one condition that must be true for the Heuristic function as seen below:
\begin{equation}
h(x,y)\leq$ Actual distance $
\end{equation}
The Heuristic function is an approximation of the cost from each cell to the goal without obstacles. An example can be seen below:
\[
\begin{bmatrix}
8 & 7 & 6 & 5 & 4\\ 
7 & 6 & 5 & 4 & 3\\ 
6 & 5 & 4 & 3 & 2\\ 
5 & 4 & 3 & 2 & 1\\ 
4 & 3 & 2 & 1 & 0
\end{bmatrix}
\]
When simulating with A\text{*} it becomes apparent that A\text{*} seeks to expand straight to the goal.

\subsection{Dynamic Programming}
Another way to plan is to use dynamic programming. When searching for the path to the goal, every spot is expanded and a policy is added to the cells. The policy is the optimal action to perform when in that cell. Actions could be to go up, down, left or right in the matrix. An example matrix is seen below: 
\[
\begin{bmatrix}
S & 1 & \downarrow & \downarrow & \downarrow\\ 
\downarrow & 1 & \downarrow & \downarrow & \downarrow\\ 
\downarrow & 1 & \rightarrow & \rightarrow & \downarrow\\ 
\rightarrow & \rightarrow & \uparrow & 1 & \downarrow\\ 
\rightarrow & \rightarrow & \uparrow & 1 & G
\end{bmatrix}
\]
Dynamic programming is useful if the robot deviates from a set path and ends up in another location. The best policy is determined by starting at the goal and going backwards. 


%------------------------------------------------
\section{Control}
\subsection{Smoothing}
The purpose of smoothing is to create a new path without very sharp turns. The new path is also more like a continuous function and very likely also shorter in length. It is important to observe the smoothed path in order to find out whether the new path will move the robot into objects placed along the route.\\
To smooth out the path we use an algorithm that retains the start and goal while smoothing points in between. The first step is to initialise $y_i = x_i$ and then optimise equation \ref{eq:optimisesmooth} with respect to equations \ref{eq:equationsforsmooth1} and \ref{eq:equationsforsmooth2}.
\begin{gather}
minimise (x_i - y_i )^2 + \alpha(y_i - y_{i+1})^2
\label{eq:optimisesmooth} \\
y_i = y_i + \alpha(x_i - y_i)
\label{eq:equationsforsmooth1} \\
y_i = y_i + \beta(y_{i+1} + y_{i-1}-2y_i)
\label{eq:equationsforsmooth2}
\end{gather}
With $0\leq\alpha\leq1$ being how much we want to retain from the original path and $0\leq\beta\leq1$ being how much we want the path smoothed. This will results in something that looks like figure \ref{fig:pathsmooth}. The green circle shows the extend of the equations being optimised at three points.
\begin{figure}[H]
\centering
\includegraphics[width=0.4\textwidth]{billeder/pathsmoothed1}
\caption{Original points and a smoothed path}
\label{fig:pathsmooth}
\end{figure}

\subsection{PID Control}
PID is short for Proportional Integral Derivative. These three parts can be seen in equation \ref{eq:pid1}. Where $e(t) = u(t) - y(t)$, $u(t)$ is the control input, $x(t)$ is the input to the motor and $y(t)$ is the output from the motor as an example.
\begin{equation}
x(t) = K_p e(t) + K_i \int^t_0 e(\tau)d\tau + K_d\dfrac{de(t)}{dt}
\label{eq:pid1}
\end{equation}
Each part of the PID controller has a function. The P controller as seen in figure \ref{fig:pcontrol} has the responsibility of reducing the rise time and decreasing the steady-state error. An overshoot appears with larger $K_p$ values. $K_p$ is analogous to a gain.
\begin{figure}[H]
\centering
\includegraphics[width=0.5\textwidth]{billeder/pcontrol}
\caption{P Controller with different $K_p$ values}
\label{fig:pcontrol}
\end{figure}
The D controller is used along with the P controller to reduce the overshoot and the settling time. The D controller uses the slope of the function to reduce the error. This is seen in figure \ref{fig:pdcontrol}.
\begin{figure}[H]
\centering
\includegraphics[width=0.5\textwidth]{billeder/pdcontrol}
\caption{PD Controller with different $K_d$ values}
\label{fig:pdcontrol}
\end{figure}
The I controller works with the integral of the error. This eliminates errors that persist over time. It will increase the overshoot and settling time, but fully eliminate the steady-state error. The is seen in figure \ref{fig:pidcontrol}.
\begin{figure}[H]
\centering
\includegraphics[width=0.5\textwidth]{billeder/pidcontrol}
\caption{PID Controller with different $K_i$ values}
\label{fig:pidcontrol}
\end{figure}
A collected table of all the effects of the three controllers can be seen in table \ref{tab:PIDcontrol}.
\begin{table}[H]
	\centering
    \begin{tabular}{|l|l|l|l|l|}
    \hline
    ~  & Rise Time    & Overshoot & Settling time & Steady-State Error \\ \hline
    $K_p$ & Decreases    & Increases & Small Change  & Decrease           \\ \hline
    $K_i$ & Decreases    & Increases & Increases     & Eliminate          \\ \hline
    $K_d$ & Small Change & Decreases & Decreases     & No Change          \\ \hline
    \end{tabular}
    \caption{PID Control effects}
    \label{tab:PIDcontrol}
\end{table}
A way to find good values for the $K_p$, $K_i$ and $K_p$ terms is to use twiddle. Twiddle or Coordinate descent works out a local minimum by doing a line search through different values of the controllers. The values can also be found in a manual way by tuning the values as you work out what is subjectively the best controller.
%------------------------------------------------
\section{SLAM}
When a robot is placed in an unknown environment, it needs to figure out where it is and how the world around it looks. For this SLAM is used, SLAM is short for Simultaneous Localization And Mapping. 
What want to achieve by using slam is to know the relative position of the robot to all seen landmarks. There is two main types of slam, online SLAM and full SLAM. online SLAM only keeps track of the relative position of the robot compared to all seen landmarks, while full SLAM keeps track of the current and all previous robot positions compared to all seen landmarks.
There are multiple SLAM algorithms, in this section, Graph SLAM will be explained shortly.

\subsection{Graph SLAM}
In the perfect world, when a robot moves, it will move exactly as intended. So if a robot is supposed to move 10 in the x direction, the new position will be $x_1 = x_0 + 10$ and $y_1 = y_0$. 
This, however is not the way it actually works. There will always be some motion noise. So instead of hitting the spot perfectly, the robot will hit the spot with some probability, which can be estimated by a Gaussian. So the new robot position will be constraint by:
\begin{equation}
f(x,y)=\frac{1}{\sqrt{2\pi\sigma^2}} \cdot e^(\frac{\frac{1}{2}(x_1-x_0-10)^2}{\sigma^2} \cdot \frac{1}{\sqrt{2\pi\sigma^2}} \cdot e^(\frac{\frac{1}{2}(y_1-y_0)^2}{\sigma^2}
\label{SLAM_eq01}
\end{equation}
This is pictured in figure \ref{SLAM_fig02}.
\begin{figure}[h!]
    \includegraphics[scale=0.5]{billeder/GraphSLAM02.png}
    \caption{A robot in a non perfect world, moving 10 in the x direction. Noise is estimated as a Gaussian with mean $\mu = [10,0]$ and standard deviation $\sigma$.}
    \label{SLAM_fig02}
\end{figure}
What we want to do is to maximise the likelihood of the new position, given the previous position.
So what Graph SLAM does is defining the probabilities using a sequence of constraints like the one in equation \ref{SLAM_eq01}. 
If a robot moves in some space, each location is characterized by a vector that in a 2 dimensional world often is 3 dimensional (it will consist of a x-coordinate, y-coordinate and an angle). 
Graph SLAM then takes the following constraints: 
\begin{itemize}
\item \textit{Location Constraint}, which is just $x_0$.
\item \textit{Relative Motion Constraints}, which relates each robot pose to the previous robot pose. These can be seen as rubber bands.
\item \textit{Relative Measurement Constraints}, which relates each robot pose with the landmarks seen from that pose. These can also be seen as rubber bands.
\end{itemize}
Graph SLAM then relaxes the set of Relative Constraints to find the most likely configuration of a robot path along with the location of landmarks.
The way to do this is to use a quadratic matrix $\Omega$ and a vector $\xi$. $\Omega$ and $\xi$ consists of the addition of all constraints. Figure \ref{SLAM_fig04} shows an example of Graph SLAM using full SLAM. Every time a constrain is set it is added to the previous constrains. In the example, the constrains are added in the end for a better understanding of the individual steps.

\begin{figure}[H]
\centering
\begin{subfigure}{.5\textwidth}
  \centering
  \includegraphics[width=.8\linewidth]{billeder/GraphSLAM04_1.png}
  \caption{Location Constrain}
  \label{SLAM_fig04:sub1}
\end{subfigure}%
\begin{subfigure}{.5\textwidth}
  \centering
  \includegraphics[width=.8\linewidth]{billeder/GraphSLAM04_2.png}
  \caption{Relative Motion Constrain from $x_0$ to $x_1$}
  \label{SLAM_fig04:sub2}
\end{subfigure}
\begin{subfigure}{.5\textwidth}
  \centering
  \includegraphics[width=.8\linewidth]{billeder/GraphSLAM04_3.png}
  \caption{Relative Measurement Constrain from $x_1$ to $L_0$}
  \label{SLAM_fig04:sub3}
\end{subfigure}%
\begin{subfigure}{.5\textwidth}
  \centering
  \includegraphics[width=.8\linewidth]{billeder/GraphSLAM04_4.png}
  \caption{Relative Motion Constrain from $x_1$ to $x_2$}
  \label{SLAM_fig04:sub4}
\end{subfigure}
\begin{subfigure}{.5\textwidth}
  \centering
  \includegraphics[width=.8\linewidth]{billeder/GraphSLAM04_5.png}
  \caption{The total Graph SLAM consists of the steps added together.}
  \label{SLAM_fig04:sub5}
\end{subfigure}
\caption{Graph SLAM illustrated.}
\label{SLAM_fig04}
\end{figure}

The most likely positions for the robot poses and the landmarks, $\mu$, are then found using equation \ref{SLAM_eq02}. 
\begin{equation}
\mu = \Omega^{-1} \cdot \xi
\label{SLAM_eq02}
\end{equation}

If noise is involved in the movements or measurements, this is handled by multiplying the constrains with $\frac{1}{\sigma}$, where $\sigma$ is the noise related to the constrain.

If the robot lives for a long time, the path taken will grow towards infinity. To avoid this, online SLAM can be used, but this will not be explained here.

%------------------------------------------------
%	Localisation
%		- Markov localisation - René
%		- Kalman - Rune
%		- Particle - Rune
%	Search and Planning - Nicolai
%	Control - Nicolai
%	SLAM - René
% chapter project
\chapter{Project}
%	Project formulation
%	Platform & Hardware
%	Design and Implementation
%		- Location
%		- Planning
%		- Driving
\section{Project formulation}
This project will entail applying knowledge from the theory chapter to handle the kidnapped robot problem. The robot that is to be used is explained in the robot platform section.\\
Some areas of the theory is needed in order to solve the kidnapped robot problem. 
These areas are localisation, path planning and movement. \\
Localisation is needed in order for the robot to locate itself within a know environment. The goal is to solve the localisation problem with a particle filter.\\
Path planning is important for the robot to be able to get to the set goal after knowing its position. The path planning problem will be solved using A\text{*}.\\
The movement will entail smoothing the path from the planner and implementing PID control to move the robot to the goal while avoiding the obstacles.\\
To properly test the implementation of these subjects, an arena will be set up with a target location and an unknown starting location. It is up to the robot to find its way home to the target location.
%\section{Platform \& Hardware}
This section contains information about the platform and hardware used in this project.


%------------------------------------------------ % Jeg tænker at alt informationen her lige så godt kunne være i design og implementation
\section{Design \& Implementation}
\lipsum[7] % Dummy text

%------------------------------------------------
% Computer (Matlab)
%		Particle filter (Runer)
%		Motion (René)
%		Planning (Nicolai)
% Arduino Uno (C code, functioncall osv) (Rune)
\section{Test}
The test setup is an arena with the dimensions 113 x 141 cm as can be seen in figure \ref{fig:testsetup}. The arena is divided into grid cells of 1x1 cm. An obsticle is placed in the middle of the arena to create a unique situation. This enables the robot to locate itself and drive towards a set goal. The map can be seen in figure \ref{fig:map}.
\begin{figure}[H]
\centering
\includegraphics[width=0.8\textwidth]{billeder/testsetup}
\caption{Test Setup}
\label{fig:testsetup}
\end{figure}
\begin{figure}[H]
\centering
\includegraphics[scale=1]{billeder/map}
\caption{Test Map}
\label{fig:map}
\end{figure}
\section{Results}
Localisation results.\\

Results when the goal is [ 20 120 ].\\

Results when the goal is [ 30 30 ].\\



%------------------------------------------------
% Resultater (René)
\chapter{Discussion}

% Motors and PID
The motors were not fitted with tachometers which meant there was no way of getting reliable information about the wheels movement. This made it hard to determine the distance driven and the drift to either side. As a direct consequence of this, PID control and path smoothing was not used as the implementation of the PID controller would have been very complex and only relying on measurements from the LIDAR sensor. It was chosen that the PID controller implementation was out of the scope of this project. \\
To try to correct some of the movement noise the LIDAR data was used to inform of the actual movement. This worked fairly well for us. If we had more time one interesting experiment would be to use a Kalman filter to fuse the measurement from the LIDAR with the time based control input to get a even better estimate. \\\\
% Planning
When a path plan is made, the plan is made from single centimetre steps into a  few coordinate sets. This is done because the robot has a hard time following the line and because of the lack of information from the wheels. When the robot is sufficiently close to one of the coordinate points, it will turn and head towards the next.\\\\
% Generelt om robottens movement (René)
From the result it can be seen that the robot uses a lot of steps to achieve its goal after the correct path is found. This is mainly because the robot is not very certain of its own position, since most steps happens when it tries to get close to the point.\\
Another problem that was often seen during the previous testing, was that the robot drifted to one side. This was somewhat solved by giving one motor a slightly higher speed than the other. The drifting error became higher the longer the robot drove in one step. Therefore the maximal distance to drive was also set to 30 cm. This gives the robot a higher precision in movements, but it also results in more movements when longer routes are planned.\\
The Move forward function was not corrected the same way as the Turn function was. If it was corrected directly by reversing, if the robot drove too far, some of the overshoots may have been avoided.\\\\
%Particle Filter 
When using the particle filter it is important to set the noise values corrected to get a optimal filter. In our filter the noise settings was good for finding the area where the true location was with relative few particles. But the noise settings prevented us for tuning in on the true location, by decreasing the area. This could be achieved by changing the noise as a function of the certainty.   
Another important aspect to consider when working with particle filters is the number of particles to be used. For a large map many particles is needed. The more particles the greater the calculation cost. It is therefore sometimes better to decrease the number of particles when the location have been found, since the very large amount is only truly needed in the initial location state. The optimal number to decrease  to is dependent of the application. \\\\
Some considerations shall also be made in regards to the resample time. In our implementation we resampled after every move. But for some applications it is a better strategy to make a series of moves and see how the probability of the particles is over time and resample based on this information. It is shown when comparing our results to results made by a other group, that have used this technique, that it preforms better in situations with many symmetric locations. This is since the resample operation is way more costly than the move operation. When during a series of movements the chance of leaving the symmetric area before a resampling increases. 


\chapter{Conclusion}
During this project a robot that are capable of locating it self and drive to a target position was created. This was archived by using the knowledge gotten from the Udacity course "Artificial Intelligence for Robotics"\cite{AIROK} and the book Probabilistic Robotics\cite{thrun2005probabilistic}. Even though many of the methods are simple in theory, applying it to a real world application proved more difficult. For a lot of the methods like particle filters to be effective, it must run i real time. This is hard to archive while having many particles, due to the calculation time. Inconsistent LIDAR data is also a challenge. In the theory we always assumed that the sensors always was working, but had some noise. In the real world we found that the LIDAR had surfaces where it did not work and sometimes random reading failed. This is just one of many special cases that must be considered when taking the theory to practice.  

When moving the robot the group found that the movement noise of a simple robot vary greatly, and is dependent on battery power, movement type and movement distance. This made it hard to use a simple noise model for the robot. In the project this was confronted by trying to minimize the noise using the LIDAR data to correct the movement. This approach worked quite well, and made it possible to assume simpler noise models.

Creating a planer, and plan a route is fairly easy. Using the A* algorithm a simple path was found. The problem is that A* finds the optimal path, and not the optimal feasible path. Some path may be the shortest but require the robot to go through openings smaller that what is possible or drive to close to the wall. In this project we used a cost map to force the path found by the planner to be feasible paths.

When following a path the robot could still bump in to obstacles if the location supplied by the particle filter imprecise. It was not uncommon that the location differed from the true location with up to a 5 cm. To ensure that the robot did not bump in to obstacles, a clear path detection system was created using the LIDAR data. This ensured that the robot would stop before diving in to a obstacle. 

The end result was a well functioning program that could drive the robot safely to the target location. 
% Discussion (Skype)
% Conclusion

%----------------------------------------------------------------------------------------
%	REFERENCE LIST
%----------------------------------------------------------------------------------------

\begingroup
	\raggedright
	\bibliography{bibtex/litteratur}							% Litteraturlisten inkluderes
\endgroup

%----------------------------------------------------------------------------------------

%\end{multicols}

\end{document}