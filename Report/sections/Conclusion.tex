\chapter{Discussion}

% Motors and PID
The motors were not fitted with tachometers which meant there was no way of getting reliable information about the wheels movement. This made it hard to determine the distance driven and the drift to either side. As a direct consequence of this, PID control and path smoothing was not used as the implementation of the PID controller would have been very complex and only relying on measurements from the LIDAR sensor. It was chosen that the PID controller implementation was out of the scope of this project. \\
To try to correct some of the movement noise the LIDAR data was used to inform of the actual movement. This worked fairly well for us. If we had more time one interesting experiment would be to use a Kalman filter to fuse the measurement from the LIDAR with the time based control input to get a even better estimate. \\\\
% Planning
When a path plan is made, the plan is made from single centimetre steps into a  few coordinate sets. This is done because the robot has a hard time following the line and because of the lack of information from the wheels. When the robot is sufficiently close to one of the coordinate points, it will turn and head towards the next.\\\\
% Generelt om robottens movement (René)
From the result it can be seen that the robot uses a lot of steps to achieve its goal after the correct path is found. This is mainly because the robot is not very certain of its own position, since most steps happens when it tries to get close to the point.\\
Another problem that was often seen during the previous testing, was that the robot drifted to one side. This was somewhat solved by giving one motor a slightly higher speed than the other. The drifting error became higher the longer the robot drove in one step. Therefore the maximal distance to drive was also set to 30 cm. This gives the robot a higher precision in movements, but it also results in more movements when longer routes are planned.\\
The Move forward function was not corrected the same way as the Turn function was. If it was corrected directly by reversing, if the robot drove too far, some of the overshoots may have been avoided.\\\\
%Particle Filter 
When using the particle filter it is important to set the noise values corrected to get a optimal filter. In our filter the noise settings was good for finding the area where the true location was with relative few particles. But the noise settings prevented us for tuning in on the true location, by decreasing the area. This could be achieved by changing the noise as a function of the certainty.   
Another important aspect to consider when working with particle filters is the number of particles to be used. For a large map many particles is needed. The more particles the greater the calculation cost. It is therefore sometimes better to decrease the number of particles when the location have been found, since the very large amount is only truly needed in the initial location state. The optimal number to decrease  to is dependent of the application. \\\\
Some considerations shall also be made in regards to the resample time. In our implementation we resampled after every move. But for some applications it is a better strategy to make a series of moves and see how the probability of the particles is over time and resample based on this information. It is shown when comparing our results to results made by a other group, that have used this technique, that it preforms better in situations with many symmetric locations. This is since the resample operation is way more costly than the move operation. When during a series of movements the chance of leaving the symmetric area before a resampling increases. 


\chapter{Conclusion}
During this project a robot that are capable of locating it self and drive to a target position was created. This was archived by using the knowledge gotten from the Udacity course "Artificial Intelligence for Robotics"\cite{AIROK} and the book Probabilistic Robotics\cite{thrun2005probabilistic}. Even though many of the methods are simple in theory, applying it to a real world application proved more difficult. For a lot of the methods like particle filters to be effective, it must run i real time. This is hard to archive while having many particles, due to the calculation time. Inconsistent LIDAR data is also a challenge. In the theory we always assumed that the sensors always was working, but had some noise. In the real world we found that the LIDAR had surfaces where it did not work and sometimes random reading failed. This is just one of many special cases that must be considered when taking the theory to practice.  

When moving the robot the group found that the movement noise of a simple robot vary greatly, and is dependent on battery power, movement type and movement distance. This made it hard to use a simple noise model for the robot. In the project this was confronted by trying to minimize the noise using the LIDAR data to correct the movement. This approach worked quite well, and made it possible to assume simpler noise models.

Creating a planer, and plan a route is fairly easy. Using the A* algorithm a simple path was found. The problem is that A* finds the optimal path, and not the optimal feasible path. Some path may be the shortest but require the robot to go through openings smaller that what is possible or drive to close to the wall. In this project we used a cost map to force the path found by the planner to be feasible paths.

When following a path the robot could still bump in to obstacles if the location supplied by the particle filter imprecise. It was not uncommon that the location differed from the true location with up to a 5 cm. To ensure that the robot did not bump in to obstacles, a clear path detection system was created using the LIDAR data. This ensured that the robot would stop before diving in to a obstacle. 

The end result was a well functioning program that could drive the robot safely to the target location. 