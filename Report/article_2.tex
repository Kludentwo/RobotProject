%%%%%%%%%%%%%%%%%%%%%%%%%%%%%%%%%%%%%%%%%
% Journal Article
% LaTeX Template
% Version 1.3 (9/9/13)
%
% This template has been downloaded from:
% http://www.LaTeXTemplates.com
%
% Original author:
% Frits Wenneker (http://www.howtotex.com)
%
% License:
% CC BY-NC-SA 3.0 (http://creativecommons.org/licenses/by-nc-sa/3.0/)
%
%%%%%%%%%%%%%%%%%%%%%%%%%%%%%%%%%%%%%%%%%

%----------------------------------------------------------------------------------------
%	PACKAGES AND OTHER DOCUMENT CONFIGURATIONS
%----------------------------------------------------------------------------------------

\documentclass[twoside]{article}

\usepackage{lipsum} % Package to generate dummy text throughout this template

\usepackage[sc]{mathpazo} % Use the Palatino font
\usepackage[T1]{fontenc} % Use 8-bit encoding that has 256 glyphs
\linespread{1.05} % Line spacing - Palatino needs more space between lines
\usepackage{microtype} % Slightly tweak font spacing for aesthetics

\usepackage[hmarginratio=1:1,top=32mm,columnsep=20pt]{geometry} % Document margins
\usepackage{multicol} % Used for the two-column layout of the document
\usepackage[hang, small,labelfont=bf,up,textfont=it,up]{caption} % Custom captions under/above floats in tables or figures
\usepackage{booktabs} % Horizontal rules in tables
\usepackage{float} % Required for tables and figures in the multi-column environment - they need to be placed in specific locations with the [H] (e.g. \begin{table}[H])
\usepackage{hyperref} % For hyperlinks in the PDF

\usepackage{lettrine} % The lettrine is the first enlarged letter at the beginning of the text
\usepackage{paralist} % Used for the compactitem environment which makes bullet points with less space between them

\usepackage{abstract} % Allows abstract customization
\renewcommand{\abstractnamefont}{\normalfont\bfseries} % Set the "Abstract" text to bold
\renewcommand{\abstracttextfont}{\normalfont\small\itshape} % Set the abstract itself to small italic text

\usepackage{titlesec} % Allows customization of titles
\renewcommand\thesection{\Roman{section}} % Roman numerals for the sections
\renewcommand\thesubsection{\Roman{subsection}} % Roman numerals for subsections
\titleformat{\section}[block]{\large\scshape\centering}{\thesection.}{1em}{} % Change the look of the section titles
\titleformat{\subsection}[block]{\large}{\thesubsection.}{1em}{} % Change the look of the section titles

\usepackage{fancyhdr} % Headers and footers
\pagestyle{fancy} % All pages have headers and footers
\fancyhead{} % Blank out the default header
\fancyfoot{} % Blank out the default footer
\fancyhead[C]{Running title $\bullet$ November 2012 $\bullet$ Vol. XXI, No. 1} % Custom header text
\fancyfoot[RO,LE]{\thepage} % Custom footer text

%----------------------------------------------------------------------------------------
%	TITLE SECTION
%----------------------------------------------------------------------------------------

\title{\vspace{-15mm}\fontsize{24pt}{10pt}\selectfont\textbf{Article Title}} % Article title

\author{
\large
\textsc{John Smith}\thanks{A thank you or further information}\\[2mm] % Your name
\normalsize University of California \\ % Your institution
\normalsize \href{mailto:john@smith.com}{john@smith.com} % Your email address
\vspace{-5mm}
}
\date{}

%----------------------------------------------------------------------------------------

\begin{document}

\maketitle % Insert title

\thispagestyle{fancy} % All pages have headers and footers

%----------------------------------------------------------------------------------------
%	ABSTRACT
%----------------------------------------------------------------------------------------

\begin{abstract}

\noindent \lipsum[1] % Dummy abstract text

\end{abstract}

%----------------------------------------------------------------------------------------
%	ARTICLE CONTENTS
%----------------------------------------------------------------------------------------

\begin{multicols}{2} % Two-column layout throughout the main article text

\section{Introduction}

\lettrine[nindent=0em,lines=3]{L} orem ipsum dolor sit amet, consectetur adipiscing elit.
\lipsum[2-3] % Dummy text

%------------------------------------------------

\section{Methods}

Maecenas sed ultricies felis. Sed imperdiet dictum arcu a egestas. 
\begin{compactitem}
\item Donec dolor arcu, rutrum id molestie in, viverra sed diam
\item Curabitur feugiat
\item turpis sed auctor facilisis
\item arcu eros accumsan lorem, at posuere mi diam sit amet tortor
\item Fusce fermentum, mi sit amet euismod rutrum
\item sem lorem molestie diam, iaculis aliquet sapien tortor non nisi
\item Pellentesque bibendum pretium aliquet
\end{compactitem}
\lipsum[4] % Dummy text

%------------------------------------------------



\input{sections\ProjectResults}
\chapter{Discussion}
The motors were not fitted with tachometers which meant there was no way of getting reliable information about the wheels movement. This made it hard to determine the distance driven and the drift to either side. As a direct consequence of this, PID control and path smoothing was not used as the implementation of the PID controller would have been very complex and only relying on measurements from the LIDAR sensor. It was chosen that the PID controller implementation was out of the scope of this project. \\
To try to correct some of the movement noise the LIDAR data was used to inform of the actual movement. This worked fairly well for us. If we had more time one interesting experiment would be to use a Kalman filter to fuse the measurement from the LIDAR with the time based control input to get a even better estimate.

When a path plan is made, the plan is made from single centimetre steps into a  few coordinate sets. 
This is done because the robot has a hard time following the line and because of the lack of information from the wheels. 
When the robot is sufficiently close to one of the coordinate points, it will turn and head towards the next.
From the result it can be seen that the robot uses a lot of steps to achieve its goal after the correct path is found. 
This is mainly because the robot is not very certain of its own position, since most steps happens when it tries to get close to the point.

Another problem that was often seen during the previous testing, was that the robot drifted to one side. This was somewhat solved by giving one motor a slightly higher speed than the other. The drifting error became higher the longer the robot drove in one step. Therefore the maximal distance to drive was also set to 30 cm. This gives the robot a higher precision in movements, but it also results in more movements when longer routes are planned.\\
The Move forward function was not corrected the same way as the Turn function was. If it was corrected directly by reversing, if the robot drove too far, some of the overshoots may have been avoided.

When using the particle filter it is important to set the noise values corrected to get a optimal filter. In our filter the noise settings was good for finding the area where the true location was with relative few particles. But the noise settings prevented us for tuning in on the true location, by decreasing the area. This could be achieved by changing the noise as a function of the certainty.   
Another important aspect to consider when working with particle filters is the number of particles to be used. For a large map many particles is needed. The more particles the greater the calculation cost. It is therefore sometimes better to decrease the number of particles when the location have been found, since the very large amount is only truly needed in the initial location state. The optimal number is dependent of the application.

Some considerations shall also be made in regards to the re-sample time. In our implementation we re-sampled after every move. But for some applications it is a better strategy to make a series of moves and see how the probability of the particles is over time and re-sample based on this information. It is shown when comparing our results to results made by a other group, that this strategy preforms better in situations with many symmetric locations. This is since the re-sample operation is way more costly than the move operation. When doing a series of movements the chance of leaving the symmetric area before a re-sampling increases. 


\chapter{Conclusion}

%----------------------------------------------------------------------------------------
%	REFERENCE LIST
%----------------------------------------------------------------------------------------

\begin{thebibliography}{99} % Bibliography - this is intentionally simple in this template

\bibitem[Figueredo and Wolf, 2009]{Figueredo:2009dg}
Figueredo, A.~J. and Wolf, P. S.~A. (2009).
\newblock Assortative pairing and life history strategy - a cross-cultural
  study.
\newblock {\em Human Nature}, 20:317--330.
 
\end{thebibliography}

%----------------------------------------------------------------------------------------

\end{multicols}

\end{document}
